\PassOptionsToPackage{utf8}{inputenc}
%\documentclass{article}

\documentclass[11pt,hidelinks]{article}
%\usepackage[margin=0.75in,top=0.75in,footskip=0.5in]{geometry}
\usepackage{comment}

\usepackage{helvet}
\renewcommand{\familydefault}{\sfdefault}

\usepackage[draft]{hyperref}
\usepackage{authblk}
\usepackage{fullpage}

\newcommand{\vocab}{\textbf}

\begin{document}

\title{Pangenome graph layout by HOGWILD! path-guided stochastic gradient descent}

\author{ %[Heumos, \textit{et~al}.]{
  Simon Heumos\,$^{1 \dagger}$,
  Andrea Guarracino\,$^{2 \dagger}$,
  Pjotr Prins\,$^3$,
  and Erik Garrison\,$^{3 *}$
}

\affil{\small$^1$Quantitative Biology Center (QBiC), University of T\"ubingen, T\"ubingen, Germany, 72076 \\
  $^2$University of Rome ``Tor Vergata'', Rome, Italy \\
  $^3$UTHSC, Memphis, TN, USA \\
  $^\ast$To whom correspondence should be addressed. \\
  $^\dagger$Contributed equally.}

\maketitle

\abstract{Pangenome graphs built from raw sets of alignments may have complex structures generated by common patterns of genome variation. These structures can introduce difficulty in downstream analyses, visualization, and interpretation. Graph sorting aims to find the best node order for a 1D and 2D layout to simplify these complex regions. Pangenome graphs embed pangenomic sequences as paths in the graph, but it remains an open challenge to consider this biological information during the sorting. Moreover, existing 2D layout methods struggle to deal with large graphs. We propose a new layout algorithm that orders the nodes of a pangenome graph using a HOGWILD! path-guided stochastic gradient descent approach: PG-SGD. Our implementation demonstrates that we can efficiently compute the latent structure of gigabase-scale pangenome graphs revealing their underlying biology.
}
\section{Introduction}

Pangenome graphs \cite{computational2016computational}.

\section{Implementation}

\section{Evaluation}

\section{Discussion}

\section*{Funding}

\bibliography{document}
\bibliographystyle{ieeetr}

\end{document}
